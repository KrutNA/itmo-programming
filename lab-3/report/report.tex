% Created 2019-11-18 Пн 02:25
% Intended LaTeX compiler: pdflatex
\documentclass[11pt]{article}
\usepackage[utf8]{inputenc}
\usepackage[T1]{fontenc}
\usepackage{graphicx}
\usepackage{grffile}
\usepackage{longtable}
\usepackage{wrapfig}
\usepackage{rotating}
\usepackage[normalem]{ulem}
\usepackage{amsmath}
\usepackage{textcomp}
\usepackage{amssymb}
\usepackage{capt-of}
\usepackage{hyperref}
\usepackage[T2A]{fontenc}
\usepackage[a4paper,left=3cm,top=2cm,right=1.5cm,bottom=2cm,marginparsep=7pt,marginparwidth=.6in]{geometry}
\usepackage{cmap}
\usepackage{xcolor}
\usepackage{listings}
\usepackage{polyglossia}
\setdefaultlanguage{russian} \setotherlanguage{english}
\setmainfont{Liberation Serif}
\setsansfont{Liberation Sans}
\setmonofont[Contextuals=Alternate,Ligatures={TeX}]{Fira Code Regular}
\author{Krutko Nikita / KrutNA}
\date{\today}
\title{}
\hypersetup{
 pdfauthor={Krutko Nikita / KrutNA},
 pdftitle={},
 pdfkeywords={},
 pdfsubject={},
 pdfcreator={Emacs 26.1 (Org mode 9.1.9)}, 
 pdflang={Russian}}
\begin{document}

\large
\thispagestyle{empty}
\begin{center}
\textbf{Национальный Исследовательский Университет ИТМО}\\
\textbf{Факультет Программной Инженерии и Компьютерной Техники}\\
\end{center}
\vspace{2em}
\begin{center}
\includegraphics[width=120pt]{itmo-logo.png}
\end{center}
\LARGE
\vspace{5em}
\begin{center}
\textbf{Вариант №84222}\\
\textbf{Лабораторная работа №3}\\
\Large
\textbf{по дисциплине}\\
\LARGE
\textbf{\emph{'Программирование'}}\\
\end{center}
\vspace{11em}
\large
\begin{flushright}
\textbf{Выполнил:}\\
\textbf{Студент группы P3113}\\
\textbf{\emph{Крутько Никита} : 242570}\\
\textbf{Преподаватель:}\\
\textbf{\emph{Письмак Алексей Евгеньевич}}\\
\end{flushright}
\vspace{4em}
\large
\begin{center}
\textbf{Санкт-Петербург 2019 г.}
\end{center}
\pagebreak{}
\setcounter{tocdepth}{2}
\tableofcontents
\vspace{2em}
\section{Задание}
\label{sec:org5ce9b12}

\subsection{Программа должна удовлетворять следующим требованиям:}
\label{sec:org4ba01f6}

\begin{enumerate}
\item Доработанная модель должна соответствовать принципам SOLID.
\item Программа должна содержать как минимум два интерфейса и один абстрактный класс (номенклатура должна быть согласована с преподавателем).
\item В разработанных классах должны быть переопределены методы equals(), toString() и hashCode().
\item Программа должна содержать как минимум один перечисляемый тип (enum).
\end{enumerate}

\subsection{Порядок выполнения работы:}
\label{sec:org1d22f3f}

\begin{enumerate}
\item Доработать объектную модель приложения.
\item Перерисовать диаграмму классов в соответствии с внесёнными в модель изменениями.
\item Согласовать с преподавателем изменения, внесённые в модель.
\item Модифицировать программу в соответствии с внесёнными в модель изменениями.
\item Отчёт по работе должен содержать:
\end{enumerate}

\subsection{Текст задания.}
\label{sec:org1746bbe}
\begin{enumerate}
\item Диаграмма классов объектной модели.
\item Исходный код программы.
\item Результат работы программы.
\item Выводы по работе.
\item Вопросы к защите лабораторной работы:
\end{enumerate}

\subsection{Принципы объектно-ориентированного программирования SOLID и STUPID.}
\label{sec:orgf6bb626}
\begin{enumerate}
\item Класс Object. Реализация его методов по умолчанию.
\item Особенности реализации наследования в Java. Простое и множественное наследование.
\item Понятие абстрактного класса. Модификатор abstract.
\item Понятие интерфейса. Реализация интерфейсов в Java, методы по умолчанию. Отличия от абстрактных классов.
\item Перечисляемый тип данных (enum) в Java. Особенности реализации и использования.
\item Методы и поля с модификаторами static и final.
\item Перегрузка и переопределение методов. Коварианты возвращаемых типов данных.
\item Элементы функционального программирования в синтаксисе Java. Функциональные интерфейсы, лямбда-выражения. Ссылки на методы.
\end{enumerate}

\subsection{Задание:}
\label{sec:orgac1a380}
\emph{Но лунатики знали, что вечно так продолжаться не может, что со временем воздух вокруг Луны совсем рассеется, отчего поверхность Луны, не защищенная значительным слоем воздуха, будет сильно прогреваться солнечными лучами и на Луне даже под стеклянным колпаком невозможно будет существовать. Вот поэтому-то лунатики стали переселяться внутрь Луны и теперь живут не с наружной, а с внутренней ее стороны, так как на самом деле Луна внутри пустая, вроде резинового мяча, и на внутренней ее поверхности можно так же прекрасно жить, как и на внешней. Эта Знайкина книжка наделала много шума. Все коротышки с увлечением читали ее. Многие ученые хвалили эту книжку за то, что она интересно написана, но все же высказывали недовольство тем, что она научно не обоснована. А действительный член академии астрономических наук профессор Звездочкин, которому тоже случилось прочитать Знайкину книжку, просто кипел от негодования и говорил, что книга эта -- вовсе не книга, а какая-то, как он выразился, чертова чепуха. Этот профессор Звездочкин был не то чтобы какой-нибудь очень сердитый субъект. Нет, он был довольно добрый коротышка, но очень, как бы это сказать, требовательный, непримиримый. Во всяком деле он ценил больше всего точность, порядок и терпеть не мог никаких фантазий, то есть выдумок.}

\section{Исходный код}
\label{sec:org86cf3e8}
github.com/KrutNA/itmo-programming/tree/lab-3/

\section{Вывод}
\label{sec:org59a9f82}
В ходе лабораторной работы я познакомился с интерфейсами, абстрактными классами и енамами. Попытался в dependency injection, не разобрался и сделал без него :с.
\end{document}