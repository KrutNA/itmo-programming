<<<<<<< HEAD
% Created 2019-11-18 Пн 02:25
=======
% Created 2020-01-14 Вт 03:45
>>>>>>> ed874cd9065882ea3d4c640af5609950f2e716f5
% Intended LaTeX compiler: pdflatex
\documentclass[11pt]{article}
\usepackage[utf8]{inputenc}
\usepackage[T1]{fontenc}
\usepackage{graphicx}
\usepackage{grffile}
\usepackage{longtable}
\usepackage{wrapfig}
\usepackage{rotating}
\usepackage[normalem]{ulem}
\usepackage{amsmath}
\usepackage{textcomp}
\usepackage{amssymb}
\usepackage{capt-of}
\usepackage{hyperref}
\usepackage[T2A]{fontenc}
\usepackage[a4paper,left=3cm,top=2cm,right=1.5cm,bottom=2cm,marginparsep=7pt,marginparwidth=.6in]{geometry}
\usepackage{cmap}
\usepackage{xcolor}
\usepackage{listings}
\usepackage{polyglossia}
\setdefaultlanguage{russian} \setotherlanguage{english}
\setmainfont{Liberation Serif}
\setsansfont{Liberation Sans}
\setmonofont[Contextuals=Alternate,Ligatures={TeX}]{Fira Code Regular}
\author{Krutko Nikita / KrutNA}
\date{\today}
\title{}
\hypersetup{
 pdfauthor={Krutko Nikita / KrutNA},
 pdftitle={},
 pdfkeywords={},
 pdfsubject={},
 pdfcreator={Emacs 26.1 (Org mode 9.1.9)}, 
 pdflang={Russian}}
\begin{document}

\large
\thispagestyle{empty}
\begin{center}
\textbf{Национальный Исследовательский Университет ИТМО}\\
\textbf{Факультет Программной Инженерии и Компьютерной Техники}\\
\end{center}
\vspace{2em}
\begin{center}
\includegraphics[width=120pt]{itmo-logo.png}
\end{center}
\LARGE
\vspace{5em}
\begin{center}
<<<<<<< HEAD
\textbf{Вариант №84222}\\
\textbf{Лабораторная работа №3}\\
=======
\textbf{Вариант №82100.147}\\
\textbf{Лабораторная работа №4}\\
>>>>>>> ed874cd9065882ea3d4c640af5609950f2e716f5
\Large
\textbf{по дисциплине}\\
\LARGE
\textbf{\emph{'Программирование'}}\\
\end{center}
\vspace{11em}
\large
\begin{flushright}
\textbf{Выполнил:}\\
\textbf{Студент группы P3113}\\
\textbf{\emph{Крутько Никита} : 242570}\\
\textbf{Преподаватель:}\\
\textbf{\emph{Письмак Алексей Евгеньевич}}\\
\end{flushright}
\vspace{4em}
\large
\begin{center}
\textbf{Санкт-Петербург 2019 г.}
\end{center}
\pagebreak{}
\setcounter{tocdepth}{2}
\tableofcontents
\vspace{2em}
\section{Задание}
<<<<<<< HEAD
\label{sec:org5ce9b12}

\subsection{Программа должна удовлетворять следующим требованиям:}
\label{sec:org4ba01f6}

\begin{enumerate}
\item Доработанная модель должна соответствовать принципам SOLID.
\item Программа должна содержать как минимум два интерфейса и один абстрактный класс (номенклатура должна быть согласована с преподавателем).
\item В разработанных классах должны быть переопределены методы equals(), toString() и hashCode().
\item Программа должна содержать как минимум один перечисляемый тип (enum).
\end{enumerate}

\subsection{Порядок выполнения работы:}
\label{sec:org1d22f3f}
=======
\label{sec:org51c8778}

\subsection{Программа должна удовлетворять следующим требованиям:}
\label{sec:orgfef1f85}
\begin{enumerate}
\item В программе должны быть реализованы 2 собственных класса исключений (checked и unchecked), а также обработка исключений этих классов.
\item В программу необходимо добавить использование локальных, анонимных и вложенных классов (static и non-static).
\end{enumerate}

\subsection{Порядок выполнения работы:}
\label{sec:org1d652d0}
>>>>>>> ed874cd9065882ea3d4c640af5609950f2e716f5

\begin{enumerate}
\item Доработать объектную модель приложения.
\item Перерисовать диаграмму классов в соответствии с внесёнными в модель изменениями.
\item Согласовать с преподавателем изменения, внесённые в модель.
\item Модифицировать программу в соответствии с внесёнными в модель изменениями.
<<<<<<< HEAD
\item Отчёт по работе должен содержать:
\end{enumerate}

\subsection{Текст задания.}
\label{sec:org1746bbe}
\begin{enumerate}
=======
\end{enumerate}

\subsection{Текст задания.}
\label{sec:org14ec5ef}

\begin{enumerate}
\item Текст задания.
>>>>>>> ed874cd9065882ea3d4c640af5609950f2e716f5
\item Диаграмма классов объектной модели.
\item Исходный код программы.
\item Результат работы программы.
\item Выводы по работе.
<<<<<<< HEAD
\item Вопросы к защите лабораторной работы:
\end{enumerate}

\subsection{Принципы объектно-ориентированного программирования SOLID и STUPID.}
\label{sec:orgf6bb626}
\begin{enumerate}
\item Класс Object. Реализация его методов по умолчанию.
\item Особенности реализации наследования в Java. Простое и множественное наследование.
\item Понятие абстрактного класса. Модификатор abstract.
\item Понятие интерфейса. Реализация интерфейсов в Java, методы по умолчанию. Отличия от абстрактных классов.
\item Перечисляемый тип данных (enum) в Java. Особенности реализации и использования.
\item Методы и поля с модификаторами static и final.
\item Перегрузка и переопределение методов. Коварианты возвращаемых типов данных.
\item Элементы функционального программирования в синтаксисе Java. Функциональные интерфейсы, лямбда-выражения. Ссылки на методы.
\end{enumerate}

\subsection{Задание:}
\label{sec:orgac1a380}
\emph{Но лунатики знали, что вечно так продолжаться не может, что со временем воздух вокруг Луны совсем рассеется, отчего поверхность Луны, не защищенная значительным слоем воздуха, будет сильно прогреваться солнечными лучами и на Луне даже под стеклянным колпаком невозможно будет существовать. Вот поэтому-то лунатики стали переселяться внутрь Луны и теперь живут не с наружной, а с внутренней ее стороны, так как на самом деле Луна внутри пустая, вроде резинового мяча, и на внутренней ее поверхности можно так же прекрасно жить, как и на внешней. Эта Знайкина книжка наделала много шума. Все коротышки с увлечением читали ее. Многие ученые хвалили эту книжку за то, что она интересно написана, но все же высказывали недовольство тем, что она научно не обоснована. А действительный член академии астрономических наук профессор Звездочкин, которому тоже случилось прочитать Знайкину книжку, просто кипел от негодования и говорил, что книга эта -- вовсе не книга, а какая-то, как он выразился, чертова чепуха. Этот профессор Звездочкин был не то чтобы какой-нибудь очень сердитый субъект. Нет, он был довольно добрый коротышка, но очень, как бы это сказать, требовательный, непримиримый. Во всяком деле он ценил больше всего точность, порядок и терпеть не мог никаких фантазий, то есть выдумок.}

\section{Исходный код}
\label{sec:org86cf3e8}
github.com/KrutNA/itmo-programming/tree/lab-3/

\section{Вывод}
\label{sec:org59a9f82}
В ходе лабораторной работы я познакомился с интерфейсами, абстрактными классами и енамами. Попытался в dependency injection, не разобрался и сделал без него :с.
=======
\end{enumerate}

\subsection{Принципы объектно-ориентированного программирования SOLID и STUPID.}
\label{sec:orgef02c93}
\begin{enumerate}
\item Обработка исключительных ситуаций, три типа исключений.
\item Вложенные, локальные и анонимные классы.
\item Механизм рефлексии (reflection) в Java. Класс Class.
\end{enumerate}

\subsection{Задание:}
\label{sec:orgffd1f8d}
\emph{И все они во главе с Кроликом пустились наутек. Пух посмотрел на свои передние лапки. Он знал, что одна из них была правая, знал он, кроме того, что если он решит, какая из них правая, то остальная будет левая. Но он никак не мог вспомнить, с чего надо начать. Они пошли. Спустя десять минут они снова остановились. Пух сказал, что заметил. Пятачок чуточку приотстал и подобрался к Пуху сзади. Когда Тигра перестал ждать, что остальные найдут его, и когда ему надоело, что рядом нет никого, кому он мог бы сказать: "Эй, пошли, что ли!"-- он подумал, что надо пойти домой. И он побежал назад. Первое, что сказала Кенга, увидав его, это: "А вот и наш милый Тигра! Как раз пора принимать рыбий жир!" И она налила ему полную чашку. Крошка Ру с гордостью заявил: "А я уже принял", и Тигра, проглотив все, что было в чашке, сказал: "И я тоже", а потом он и Ру стали дружески толкать друг друга, и Тигра случайно перевернул один или два стула, нечаянно, а Крошка Ру случайно перевернул один, нарочно, и Кенга сказала: И они послушно отправились к Шести Соснам и стали кидать друг в друга шишками, и за этим приятным занятием забыли, зачем они пришли в Лес, и забыли заодно корзинку под деревом, а сами отправились домой обедать. Обед как раз подходил к концу, когда Кристофер Робин заглянул в дверь. Тигра стал объяснять, что произошло, и в то же самое время Крошка Ру стал объяснять про свой бисквитный Кашель, а Кенга стала уговаривать их не говорить одновременно. Так что прошло немало времени, пока Кристофер Робин понял, что Пух, Пятачок и Кролик бродят где-то в тумане, заблудившись в Лесной Чаще. Все трое отдыхали в маленькой ямке с песком. Пуху ужасно надоела эта ямка с песком, и он серьезно подозревал, что она просто-таки бегает за ними по пятам, потому что, куда бы они ни направились, они обязательно натыкались на нее. Каждый раз, когда она появлялась из тумана, Кролик торжествующе заявлял: "Теперь я знаю, где мы!", а Пух грустно говорил: "Я тоже".}

\section{Исходный код}
\label{sec:orge77b9ca}
se.ifmo.ru/\textasciitilde{}s242570/labs/prog/4/

\section{Вывод}
\label{sec:org321da9a}
В ходе лабораторной работы я изучил механизмы рефлексии, использовал совсем не нужные в контексте данной задачи локальные классы, сделал SOLID'но.
>>>>>>> ed874cd9065882ea3d4c640af5609950f2e716f5
\end{document}