% Created 2019-09-28 Сб 21:01
% Intended LaTeX compiler: pdflatex
\documentclass[11pt]{article}
\usepackage[utf8]{inputenc}
\usepackage[T1]{fontenc}
\usepackage{graphicx}
\usepackage{grffile}
\usepackage{longtable}
\usepackage{wrapfig}
\usepackage{rotating}
\usepackage[normalem]{ulem}
\usepackage{amsmath}
\usepackage{textcomp}
\usepackage{amssymb}
\usepackage{capt-of}
\usepackage{hyperref}
\usepackage[T2A]{fontenc}
\usepackage[a4paper,left=3cm,top=2cm,right=1.5cm,bottom=2cm,marginparsep=7pt,marginparwidth=.6in]{geometry}
\usepackage{cmap}
\usepackage{xcolor}
\usepackage{listings}
\usepackage{polyglossia}
\setdefaultlanguage{russian} \setotherlanguage{english}
\setmainfont{Liberation Serif}
\setsansfont{Liberation Sans}
\setmonofont[Contextuals=Alternate,Ligatures={TeX}]{Fira Code Regular}
\author{Крутько Никита}
\date{\today}
\title{}
\hypersetup{
 pdfauthor={Крутько Никита},
 pdftitle={},
 pdfkeywords={},
 pdfsubject={},
 pdfcreator={Emacs 26.1 (Org mode 9.1.9)}, 
 pdflang={Russian}}
\begin{document}

\begin{center}
\textbf{Санкт-Петербургский Национальный Исследовательский}\\
\textbf{Университет Информационных Технологий, Механики и Оптики}\\
\textbf{Факультет Программной Инженерии и Компьютерной Техники}\\
\end{center}
\vspace{1em}
\begin{center}
\includegraphics[width=120pt]{./itmo-logo.png}
\end{center}
\LARGE
\vspace{5em}
\begin{center}
\textbf{Вариант №824714}\\
\textbf{Лабораторная работа №1}\\
\Large
\textbf{по дисциплине}\\
\LARGE
\textbf{\emph{'Программирование'}}\\
\end{center}
\vspace{11em}
\large
\begin{flushright}
\textbf{Выполнил:}\\
\textbf{Студент группы P3113}\\
\textbf{\emph{Крутько Никита} : 242570}\\
\textbf{Преподаватель:}\\
\textbf{\emph{Письмак Алексей Евгеньевич}}\\
\end{flushright}
\vspace{4em}
\large
\begin{center}
\textbf{Санкт-Петербург 2019 г.}
\end{center}
\pagebreak{}
\section{Задание:}
\label{sec:org089fcbd}
Написать программу на языке \textbf{Java}, выполняющую соответствующие варианту действия. Программа должна соответствовать следующим требованиям:
\begin{enumerate}
\item Она должна быть упакована в исполняемый jar-архив.
\item Выражение должно вычисляться в соответствии с правилами вычисления математи\-чес\-ких выражений (должен соблюдаться порядок выполнения действий и т.д.).
\item Программа должна использовать математические функции из стандартной биб\-лиотеки Java.
\item Результат вычисления выражения должен быть выведен в стандартный поток вывода в заданном формате.
\end{enumerate}
Выполнение программы необходимо продемонстрировать на сервере \textbf{hellios}.

\section{Исходный код:}
\label{sec:org570de23}
\small
\lstset{language=Java,label= ,caption={Main.java},captionpos=b,numbers=none}
\begin{lstlisting}
import java.lang.Math;
import java.util.stream.IntStream;

public class Main {

    public static void main(String[] args) {

	// Setup constants for first array
	final int firstArraySize = 10;
	// Setup constants for second array
	final int secondArraySize = 10;
	final double secondArrayMin = -5.0;
	final double secondArrayMax = 9.0;
	// Setup constants for third array
	final int firstFunctionExpectedValue = 11;
	final int[] secondFunctionExpectedArray = {
	    7, 15, 17, 21, 23
	};
	Main.
	// Arrays initialization
	long[] first = new long[firstArraySize];
	double[] second = new double[secondArraySize];
	double[][] third = new double[firstArraySize][secondArraySize];
	System.
	// Setup values of first array
	for (int i = 0; i < firstArraySize; ++i) {
	    first[i] = setupFirstArray(i);
	}
	// Setup values of second array
	for (int i = 0; i < secondArraySize; ++i) {
	    second[i] = (double)
		(Math.random() *
		 (secondArrayMax - secondArrayMin) +
		 secondArrayMin);

	}
	// Setup values of third array
	for (int i = 0; i < firstArraySize; ++i) {
	    for (int j = 0; j < secondArraySize; ++j) {
		final long value = first[i];
		final double result = second[j];
		if (value == firstFunctionExpectedValue) {

		    third[i][j] = executeFirstFunction(result);

		} else if (IntStream.of(secondFunctionExpectedArray)
			   .anyMatch(val -> val == value)) {

		    third[i][j] = executeSecondFunction(result);

		} else {

		    third[i][j] = executeThirdFunction(result);

		}
	    }
	}
	// Print third function to the screen
	printTwoDimensionalArrayToScreen(third, firstArraySize, secondArraySize);
	// printTwoDimensionalArrayToScreenWithTable(third,
	// 		  first, firstArraySize,
	// 		  second, secondArraySize);
    }

    static long setupFirstArray
	(final int value) {

	return (value << 1) + 5; 
    }

    static double executeFirstFunction
	(final double value) {

	return Math.atan
	    (Math.cos
	     (Math.log
	      (powAnyValue
	       (Math.sin(value), 2.0D))));
    }

    static double executeSecondFunction
	(final double value) {

	return Math.tan
	    (Math.cos
	     (Math.asin
	      ((value + 2.0D) / 14.0D)));
    }

    static double executeThirdFunction
	(final double value) {

	return Math.asin
	    (Math.sin
	     (Math.asin
	      (Math.sin
	       (Math.log
		(powAnyValue
		 (2.0D / Math.abs(value), value))))));
    }

    static double powAnyValue
	(double value,
	 final double dimension) {

	boolean isNegative = value < 0 && Math.pow(value, dimension) == Double.NaN;
	if (isNegative) {
	    value *= -1.0;
	}
	double result = Math.pow(value, dimension);
	if (isNegative) {
	    result *= -1.0;
	}
	return result;
    }

    private static void printTwoDimensionalArrayToScreen
	(final double[][] arr,
	 final int columns,
	 final int lines) {

	for (int i = 0; i < columns; ++i) {
	    for (int j = 0; j < lines; ++j) {
		System.out.printf("%13.4e", arr[i][j]);
	    }
	    System.out.println();
	}
    }
    private static void printTwoDimensionalArrayToScreenWithTable
	(final double[][] arr,
	 final long[] first,
	 final int lines,
	 final double[] second,
	 final int columns) {

	System.out.printf("    \u2503");
	for (int i = 0; i < columns; ++i) {
	    System.out.printf("%13.4e", second[i]);
	}
	System.	out.println();
	for (int i = 0; i < columns * 13 + 5; ++i) {
	    if (i == 4)
		System.out.printf("\u254B");
	    else
		System.out.printf("\u2501");
	}
	System.out.println();
	for (int i = 0; i < lines; ++i) {
	    System.out.printf("%3d \u2503", first[i]);
	    for (int j = 0; j < columns; ++j) {
		System.out.printf("%13.4e", arr[i][j]);
	    }
	    System.out.println();
	}
    }

}
\end{lstlisting}
\section{Вывод программы:}
\label{sec:org7f283e3}
\fontsize{6}{8} \selectfont
\begin{verbatim}
1,0247e+00   1,5154e+00  -1,0988e+00  -1,7137e-01  -9,8037e-01   7,6337e-01   1,0950e+00   1,6724e-02  -7,6277e-01   2,5500e-01
1,1412e+00   8,4987e-01   1,3588e+00   1,4124e+00   1,0389e+00   8,0466e-01   1,5501e+00   1,5574e+00   1,0500e+00   1,4404e+00
1,0247e+00   1,5154e+00  -1,0988e+00  -1,7137e-01  -9,8037e-01   7,6337e-01   1,0950e+00   1,6724e-02  -7,6277e-01   2,5500e-01
7,3699e-01   7,8539e-01  -7,7968e-01   7,4927e-01   7,1027e-01   7,7508e-01  -7,7770e-01   7,7474e-01   5,4438e-01   7,8525e-01
1,0247e+00   1,5154e+00  -1,0988e+00  -1,7137e-01  -9,8037e-01   7,6337e-01   1,0950e+00   1,6724e-02  -7,6277e-01   2,5500e-01
1,1412e+00   8,4987e-01   1,3588e+00   1,4124e+00   1,0389e+00   8,0466e-01   1,5501e+00   1,5574e+00   1,0500e+00   1,4404e+00
1,1412e+00   8,4987e-01   1,3588e+00   1,4124e+00   1,0389e+00   8,0466e-01   1,5501e+00   1,5574e+00   1,0500e+00   1,4404e+00
1,0247e+00   1,5154e+00  -1,0988e+00  -1,7137e-01  -9,8037e-01   7,6337e-01   1,0950e+00   1,6724e-02  -7,6277e-01   2,5500e-01
1,1412e+00   8,4987e-01   1,3588e+00   1,4124e+00   1,0389e+00   8,0466e-01   1,5501e+00   1,5574e+00   1,0500e+00   1,4404e+00
1,1412e+00   8,4987e-01   1,3588e+00   1,4124e+00   1,0389e+00   8,0466e-01   1,5501e+00   1,5574e+00   1,0500e+00   1,4404e+00
\end{verbatim}
\large
\section{Вывод:}
\label{sec:org170612b}
Я познакомился с языком \textbf{Java}, изучил некоторые библиотеки (\textbf{Math}), методы других классов, основные примитивные типы данных языка и их особенности. \\
Изучил некоторые основные параметры javac, java, jar, jdb и использовал их для сборки и написания программы. \\
Также изучил JDK и JRE: какие основные компоненты в них входят и их назначение.
\end{document}